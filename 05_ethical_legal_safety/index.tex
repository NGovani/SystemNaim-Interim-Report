\chapter{Ethical, Legal \& Safety Plan}

\section{Ethical Plan}

The vast majority of the project has no ethical concerns, except for the user difficulty evaluation outlined in \autoref{sec:ud}. If a sample people were used to evaluate the project then information gathering would have to take place resulting in some privacy concerns. Fortunately, for this task sensitive information is not necessary, such as name or date of birth, instead the sole focus would be on how long that individual too to design a project and also a qualitative response from them about their experience. 

\section{Legal Plan}

Legal concerns for this project mainly surround plagiarism of existing solutions. However, most solutions do not reveal anything about their inner working as is the case for Vivado HLS, LegUp or Intel High Level Compiler, therefore it would be extremely difficult to intentionally plagiarise any work. Also, all current design choices, outlined in \autoref{chap:Design}, are either using public standards such as Ethernet or common methodologies like RPC.

Another concern would be using Intel owned IP's without proper credit. According to \cite{intel-license} using IP's in evaluation mode requires no licence and is the mode the project will be operating under for its entire duration. If the product were to ever develop into production, licences would then need to be bought for proper legal use.

\section{Safety Plan}

All equipment used in this project, are connected using standard interfaces such as USB or Ethernet. As such, there are no safety concerns for bad soldering or other hazardous techniques. Alongside this, the majority of the development will occur in software resulting in very few safety risks besides bad posture or eye strain.