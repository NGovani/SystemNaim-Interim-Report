\chapter{Project Specification}

\section{Aims}

The aim of System-Naim is to create a HLS tool which allows users to target a multi-FPGA system. Most modern HLS tools, which will be touched on later, require the user to manually specify the split of the system and implement their own communication protocol. This project intends to alleviate the effort that must be put in by the user by automating much of the process.

\section{Motivation}

Before we go into detail about the specifications of the tool, we must first understand the necessity of it. This section explores why FPGAs, HLS tools and multi-FPGA systems are each important areas of research and gives insight into the motivation for this project.

\subsection{FPGAs}

FPGAs(Field-Programmable Gate Arrays) have been around for since the 1980s and have served various purposes over the years\cite{fpga-history}. They boast a faster time-to-market than ASICs whilst still providing a performance boost when compared to software running on a GPPs(General purpose processor). These features, amongst others, make FPGAs favourable for a variety of uses including:
\begin{itemize}
    \item Prototyping.
    \item Hardware Emulation.
    \item Data processing for data centres.
    \item High-frequency trading.
 \end{itemize}

In more technical terms an FPGA is a large collection of LE(Logic Elements) which can programmed using an HDL(Hardware Description Language) such as Verilog. Each of these LE can only perform a small amount of logic such as implementing an AND/OR gate but an average FPGA comes with tens of thousands of them allowing for large amounts of logic to be strung together. The important fact is that, once programmed, an FPGA becomes a piece of hardware dedicated to performing a fixed operation and thus, if programmed correctly, gives lower latencies and higher-throughputs alongside lower power requirements than the same operation being conducted in software on a GPP. In summary, if a user is looking for a way to decrease the latency, or increase the throughput, of a rapidly changing system, an FPGA would definitely be a strong contender for a solution. 

\subsection{HLS Tools}

Notice however, that in the section above it was explicitly mentioned that benefits of using an FPGA only came if the device in question is programmed correctly. Now correctness in this context refers to multiple things. Firstly, the operation must be suited to an FPGA. This can include data access patterns or whether the application has a large amount of parallelism the programmer can exploit, but the key takeaway is that not all programs are suited to being executed on an FPGA and some may perform better on a GPP, and it is up to the designer/programmer to make this judgement call.

In this section, however, we are more concerned about another aspect of correctness; the difficulty of writing HDL code. FPGA development is notoriously difficult due to the lack of individuals with the skills to write HDL code targeted at FPGAs. The cause of this issue is two-fold, one is that FPGA usage isn't very well known and the second being the paradigm shift between standard programming, such as in languages like C++ or python, and programming with an HDL. An entire report could be written to explain the differences between hardware and software programming, but the key point is that it's very difficult to shift between the two which results in fewer people who have the skills to exploit the benefits of an FPGA effectively. To help solve this problem HLS(High-Level Synthesis) tools have been researched and created both by academics and industry leader. From Xilinx we have Vivado and from Intel we have OpenCL.

These tools allow users to write code in C++/C which is then transpiled to Verilog for use on an FPGA. In order to create optimised code these tools include configuration options and a feature known as PRAGMAS, which allow the programmer more control over how their code is turned into hardware, this addition does mean there is some barrier to entry when using these tools however it is much less than going from C++ to Verilog. 

Overall these tools allows for a much easier transfer of skills from normal programming to hardware design, paving the way for a larger part of industry and academia to gain the potential benefits of using an FPGA. Therefore, research into making these tools more accessible and on par, performance wise, with pure Verilog design is very appealing when one considers the rising usage FPGAs in variety of fields\cite{fpga-market}.

\subsection{Multi-FPGA Communication}

One of the major limitations of any hardware implementation is the resources available to a programmer on the device. Examples of such resources are:

\begin{itemize}
    \item \textbf{BRAMs}: For on-chip storage.
    \item \textbf{DRAM Bandwidth}: For accessing off-chip memory.
    \item \textbf{DSPs}: For floating-point arithmetic.
    \item \textbf{LUTs}: For logic processing.
\end{itemize}

If the programmer finds themselves having exhausted the resources on the board but still in need of more, they have a few options that they can explore. The first and usually the most practical is to re-structure the design. This may include adding pipeline stages to remove timing violations, moving to data storage off-chip to increase on-chip memory space or reusing modules in multiple parts of the design to save on LUTs. All optimisations while viable do have downsides and considerations the programmer must take in to account before proceeding. The next option would be to get a device with more resources available, while ideal this option may not be possible either because the new device would be too expensive or more likely the programmer has been given a specification and a part of it is to use this specific device. The third option and the one of interest to this report is to use multiple FPGAs in parallel to compute different parts of the system giving the programmer access to more resources in doing so.

Utilising a multi-FPGA system to overcome the limitations of a single FPGA has been the focal point for many research fronts including: DNN Inference\cite{10.1145/3358192}, Energy Conversion System Simulation\cite{8822485}, Large Graph Processing\cite{10.1145/3020078.3021739}, and CNN Acceleration \cite{10.1145/3337821.3337846} and thus proves there is benefit to allowing more people access to such systems. There are of course drawbacks for this method such as the overhead communication cost between the two FPGAs and those will be explored later in the report.

\section{Deliverables}

This project consists mainly of two deliverables, one is the HLS tool, System-Naim, which is entirely based in software and the other is an analytical report of the tool itself. This report will evaluate the efficacy of the System-Naim as well as indicate areas where further development would be beneficial.

\subsection{System-Naim}

As previously stated, the aim of System-Naim is to create a HLS tool which allows users to target a multi-FPGA system. However, we must acknowledge that this being an FYP there a time constraint which will limit the amount of work possible. Therefore, some limitations have been placed on the scope of System-Naim in order to ensure that a working tool can be produced. They are as follows:

\begin{itemize}
    \item The input will be a single source file following a subset of ANSI C90. The limitations of the language are:
    \begin{itemize}
        \item Only integer types will be allowed.
        \item Only local arrays will be allowed.
        \item Structs \& enums will not be allowed.
    \end{itemize}
    \item Only two FPGAs will communicate with each other.
\end{itemize}

It should be noted that these limitations are subject to change and certain design choices will introduce more. Also, if work on the base tool is completed sooner than expected, some of these limitations will be removed and work will be done to increase scope of the tool. If this is the case then the two expansions that would be of most interest are the allowing of passing arrays between functions and more than two FPGAs communicating.

The output of System-Naim will be a collection of Verilog HDL files which perform the same computation as the provided input file and with each FPGA receiving its own top file. The target devices have yet to be decided, however, it is likely that regardless of this choice the output files will simply be accompanied by a guide on how to implement the design using the currently available synthesis tools. It is also yet to be decided whether pre-existing libraries for base functionality, such as arithmetic and memory access, will be used. This choice will be made after the target device is chosen and depends upon the difficulty of incorporating these libraries into the tool.


\subsection{Analytical Report}

The analysis report will 

