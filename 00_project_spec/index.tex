\chapter{Project Specification}

\section{Introduction}
\subsection{FPGAs}
FPGAs(Field-Programmable Gate Arrays) have been around for since the the 1980s and have served various purposes over the years. They boast a faster time-to-market than ASICs whilst still providing a performance boost when compared to software running on a GPPs(General purpose processor). These features, amongst others, make FPGAs favourable for a variety of uses including:
\begin{itemize}
    \item Prototyping
    \item Hardware Emulation
    \item Data processing for data centres
    \item High-frequency trading
 \end{itemize}

In more technical terms an FPGA is a large collection of LEs(Logic Elements) which can programmed using a HDL(Hardware Description Language) such as Verilog. Each of these LEs can only perform a small amount of logic such as implementing an AND/OR gate but an average FPGA comes with tens of thousands of them allowing for large amounts of logic to be strung together. The important fact is that, once programmed, an FPGA becomes a piece of hardware dedicated to performing a fixed operation and thus, if programmed correctly, gives lower latencies and higher-throughputs alongside lower power requirements than the same operation being conducted in software on a GPP. To summarise, if a user is looking for a way to decrease the latency, or increase the throughput, of a rapidly changing system, an FPGA would definitely be a strong contender for a solution. 

\subsection{HLS Tools}
Notice however, that in my section above I explicitly mentioned that benefits of using an FPGA only came if the device in question is programmed correctly. Now correctness in this context refers to multiple things. Firstly, the operation must be suited to an FPGA. This can include data access patterns or whether or not the application has a large amount of parallelism the programmer can exploit, but the key takeaway is that not all programs are suited to being executed on an FPGA and some may perform better on a GPP and it is up to the designer/programmer to make this judgement call.

However, in this report we are more concerned about another aspect of correctness; the difficulty of creating optimised HDL code.



 